
% Default to the notebook output style

    


% Inherit from the specified cell style.




    
\documentclass[11pt]{article}

    
    
    \usepackage[T1]{fontenc}
    % Nicer default font (+ math font) than Computer Modern for most use cases
    \usepackage{mathpazo}

    % Basic figure setup, for now with no caption control since it's done
    % automatically by Pandoc (which extracts ![](path) syntax from Markdown).
    \usepackage{graphicx}
    % We will generate all images so they have a width \maxwidth. This means
    % that they will get their normal width if they fit onto the page, but
    % are scaled down if they would overflow the margins.
    \makeatletter
    \def\maxwidth{\ifdim\Gin@nat@width>\linewidth\linewidth
    \else\Gin@nat@width\fi}
    \makeatother
    \let\Oldincludegraphics\includegraphics
    % Set max figure width to be 80% of text width, for now hardcoded.
    \renewcommand{\includegraphics}[1]{\Oldincludegraphics[width=.8\maxwidth]{#1}}
    % Ensure that by default, figures have no caption (until we provide a
    % proper Figure object with a Caption API and a way to capture that
    % in the conversion process - todo).
    \usepackage{caption}
    \DeclareCaptionLabelFormat{nolabel}{}
    \captionsetup{labelformat=nolabel}

    \usepackage{adjustbox} % Used to constrain images to a maximum size 
    \usepackage{xcolor} % Allow colors to be defined
    \usepackage{enumerate} % Needed for markdown enumerations to work
    \usepackage{geometry} % Used to adjust the document margins
    \usepackage{amsmath} % Equations
    \usepackage{amssymb} % Equations
    \usepackage{textcomp} % defines textquotesingle
    % Hack from http://tex.stackexchange.com/a/47451/13684:
    \AtBeginDocument{%
        \def\PYZsq{\textquotesingle}% Upright quotes in Pygmentized code
    }
    \usepackage{upquote} % Upright quotes for verbatim code
    \usepackage{eurosym} % defines \euro
    \usepackage[mathletters]{ucs} % Extended unicode (utf-8) support
    \usepackage[utf8x]{inputenc} % Allow utf-8 characters in the tex document
    \usepackage{fancyvrb} % verbatim replacement that allows latex
    \usepackage{grffile} % extends the file name processing of package graphics 
                         % to support a larger range 
    % The hyperref package gives us a pdf with properly built
    % internal navigation ('pdf bookmarks' for the table of contents,
    % internal cross-reference links, web links for URLs, etc.)
    \usepackage{hyperref}
    \usepackage{longtable} % longtable support required by pandoc >1.10
    \usepackage{booktabs}  % table support for pandoc > 1.12.2
    \usepackage[inline]{enumitem} % IRkernel/repr support (it uses the enumerate* environment)
    \usepackage[normalem]{ulem} % ulem is needed to support strikethroughs (\sout)
                                % normalem makes italics be italics, not underlines
    

    
    
    % Colors for the hyperref package
    \definecolor{urlcolor}{rgb}{0,.145,.698}
    \definecolor{linkcolor}{rgb}{.71,0.21,0.01}
    \definecolor{citecolor}{rgb}{.12,.54,.11}

    % ANSI colors
    \definecolor{ansi-black}{HTML}{3E424D}
    \definecolor{ansi-black-intense}{HTML}{282C36}
    \definecolor{ansi-red}{HTML}{E75C58}
    \definecolor{ansi-red-intense}{HTML}{B22B31}
    \definecolor{ansi-green}{HTML}{00A250}
    \definecolor{ansi-green-intense}{HTML}{007427}
    \definecolor{ansi-yellow}{HTML}{DDB62B}
    \definecolor{ansi-yellow-intense}{HTML}{B27D12}
    \definecolor{ansi-blue}{HTML}{208FFB}
    \definecolor{ansi-blue-intense}{HTML}{0065CA}
    \definecolor{ansi-magenta}{HTML}{D160C4}
    \definecolor{ansi-magenta-intense}{HTML}{A03196}
    \definecolor{ansi-cyan}{HTML}{60C6C8}
    \definecolor{ansi-cyan-intense}{HTML}{258F8F}
    \definecolor{ansi-white}{HTML}{C5C1B4}
    \definecolor{ansi-white-intense}{HTML}{A1A6B2}

    % commands and environments needed by pandoc snippets
    % extracted from the output of `pandoc -s`
    \providecommand{\tightlist}{%
      \setlength{\itemsep}{0pt}\setlength{\parskip}{0pt}}
    \DefineVerbatimEnvironment{Highlighting}{Verbatim}{commandchars=\\\{\}}
    % Add ',fontsize=\small' for more characters per line
    \newenvironment{Shaded}{}{}
    \newcommand{\KeywordTok}[1]{\textcolor[rgb]{0.00,0.44,0.13}{\textbf{{#1}}}}
    \newcommand{\DataTypeTok}[1]{\textcolor[rgb]{0.56,0.13,0.00}{{#1}}}
    \newcommand{\DecValTok}[1]{\textcolor[rgb]{0.25,0.63,0.44}{{#1}}}
    \newcommand{\BaseNTok}[1]{\textcolor[rgb]{0.25,0.63,0.44}{{#1}}}
    \newcommand{\FloatTok}[1]{\textcolor[rgb]{0.25,0.63,0.44}{{#1}}}
    \newcommand{\CharTok}[1]{\textcolor[rgb]{0.25,0.44,0.63}{{#1}}}
    \newcommand{\StringTok}[1]{\textcolor[rgb]{0.25,0.44,0.63}{{#1}}}
    \newcommand{\CommentTok}[1]{\textcolor[rgb]{0.38,0.63,0.69}{\textit{{#1}}}}
    \newcommand{\OtherTok}[1]{\textcolor[rgb]{0.00,0.44,0.13}{{#1}}}
    \newcommand{\AlertTok}[1]{\textcolor[rgb]{1.00,0.00,0.00}{\textbf{{#1}}}}
    \newcommand{\FunctionTok}[1]{\textcolor[rgb]{0.02,0.16,0.49}{{#1}}}
    \newcommand{\RegionMarkerTok}[1]{{#1}}
    \newcommand{\ErrorTok}[1]{\textcolor[rgb]{1.00,0.00,0.00}{\textbf{{#1}}}}
    \newcommand{\NormalTok}[1]{{#1}}
    
    % Additional commands for more recent versions of Pandoc
    \newcommand{\ConstantTok}[1]{\textcolor[rgb]{0.53,0.00,0.00}{{#1}}}
    \newcommand{\SpecialCharTok}[1]{\textcolor[rgb]{0.25,0.44,0.63}{{#1}}}
    \newcommand{\VerbatimStringTok}[1]{\textcolor[rgb]{0.25,0.44,0.63}{{#1}}}
    \newcommand{\SpecialStringTok}[1]{\textcolor[rgb]{0.73,0.40,0.53}{{#1}}}
    \newcommand{\ImportTok}[1]{{#1}}
    \newcommand{\DocumentationTok}[1]{\textcolor[rgb]{0.73,0.13,0.13}{\textit{{#1}}}}
    \newcommand{\AnnotationTok}[1]{\textcolor[rgb]{0.38,0.63,0.69}{\textbf{\textit{{#1}}}}}
    \newcommand{\CommentVarTok}[1]{\textcolor[rgb]{0.38,0.63,0.69}{\textbf{\textit{{#1}}}}}
    \newcommand{\VariableTok}[1]{\textcolor[rgb]{0.10,0.09,0.49}{{#1}}}
    \newcommand{\ControlFlowTok}[1]{\textcolor[rgb]{0.00,0.44,0.13}{\textbf{{#1}}}}
    \newcommand{\OperatorTok}[1]{\textcolor[rgb]{0.40,0.40,0.40}{{#1}}}
    \newcommand{\BuiltInTok}[1]{{#1}}
    \newcommand{\ExtensionTok}[1]{{#1}}
    \newcommand{\PreprocessorTok}[1]{\textcolor[rgb]{0.74,0.48,0.00}{{#1}}}
    \newcommand{\AttributeTok}[1]{\textcolor[rgb]{0.49,0.56,0.16}{{#1}}}
    \newcommand{\InformationTok}[1]{\textcolor[rgb]{0.38,0.63,0.69}{\textbf{\textit{{#1}}}}}
    \newcommand{\WarningTok}[1]{\textcolor[rgb]{0.38,0.63,0.69}{\textbf{\textit{{#1}}}}}
    
    
    % Define a nice break command that doesn't care if a line doesn't already
    % exist.
    \def\br{\hspace*{\fill} \\* }
    % Math Jax compatability definitions
    \def\gt{>}
    \def\lt{<}
    % Document parameters
    \title{NOTEBOOK\_TP\_IMAGE\_GIBERT\_ALEXIS}
    
    
    

    % Pygments definitions
    
\makeatletter
\def\PY@reset{\let\PY@it=\relax \let\PY@bf=\relax%
    \let\PY@ul=\relax \let\PY@tc=\relax%
    \let\PY@bc=\relax \let\PY@ff=\relax}
\def\PY@tok#1{\csname PY@tok@#1\endcsname}
\def\PY@toks#1+{\ifx\relax#1\empty\else%
    \PY@tok{#1}\expandafter\PY@toks\fi}
\def\PY@do#1{\PY@bc{\PY@tc{\PY@ul{%
    \PY@it{\PY@bf{\PY@ff{#1}}}}}}}
\def\PY#1#2{\PY@reset\PY@toks#1+\relax+\PY@do{#2}}

\expandafter\def\csname PY@tok@w\endcsname{\def\PY@tc##1{\textcolor[rgb]{0.73,0.73,0.73}{##1}}}
\expandafter\def\csname PY@tok@c\endcsname{\let\PY@it=\textit\def\PY@tc##1{\textcolor[rgb]{0.25,0.50,0.50}{##1}}}
\expandafter\def\csname PY@tok@cp\endcsname{\def\PY@tc##1{\textcolor[rgb]{0.74,0.48,0.00}{##1}}}
\expandafter\def\csname PY@tok@k\endcsname{\let\PY@bf=\textbf\def\PY@tc##1{\textcolor[rgb]{0.00,0.50,0.00}{##1}}}
\expandafter\def\csname PY@tok@kp\endcsname{\def\PY@tc##1{\textcolor[rgb]{0.00,0.50,0.00}{##1}}}
\expandafter\def\csname PY@tok@kt\endcsname{\def\PY@tc##1{\textcolor[rgb]{0.69,0.00,0.25}{##1}}}
\expandafter\def\csname PY@tok@o\endcsname{\def\PY@tc##1{\textcolor[rgb]{0.40,0.40,0.40}{##1}}}
\expandafter\def\csname PY@tok@ow\endcsname{\let\PY@bf=\textbf\def\PY@tc##1{\textcolor[rgb]{0.67,0.13,1.00}{##1}}}
\expandafter\def\csname PY@tok@nb\endcsname{\def\PY@tc##1{\textcolor[rgb]{0.00,0.50,0.00}{##1}}}
\expandafter\def\csname PY@tok@nf\endcsname{\def\PY@tc##1{\textcolor[rgb]{0.00,0.00,1.00}{##1}}}
\expandafter\def\csname PY@tok@nc\endcsname{\let\PY@bf=\textbf\def\PY@tc##1{\textcolor[rgb]{0.00,0.00,1.00}{##1}}}
\expandafter\def\csname PY@tok@nn\endcsname{\let\PY@bf=\textbf\def\PY@tc##1{\textcolor[rgb]{0.00,0.00,1.00}{##1}}}
\expandafter\def\csname PY@tok@ne\endcsname{\let\PY@bf=\textbf\def\PY@tc##1{\textcolor[rgb]{0.82,0.25,0.23}{##1}}}
\expandafter\def\csname PY@tok@nv\endcsname{\def\PY@tc##1{\textcolor[rgb]{0.10,0.09,0.49}{##1}}}
\expandafter\def\csname PY@tok@no\endcsname{\def\PY@tc##1{\textcolor[rgb]{0.53,0.00,0.00}{##1}}}
\expandafter\def\csname PY@tok@nl\endcsname{\def\PY@tc##1{\textcolor[rgb]{0.63,0.63,0.00}{##1}}}
\expandafter\def\csname PY@tok@ni\endcsname{\let\PY@bf=\textbf\def\PY@tc##1{\textcolor[rgb]{0.60,0.60,0.60}{##1}}}
\expandafter\def\csname PY@tok@na\endcsname{\def\PY@tc##1{\textcolor[rgb]{0.49,0.56,0.16}{##1}}}
\expandafter\def\csname PY@tok@nt\endcsname{\let\PY@bf=\textbf\def\PY@tc##1{\textcolor[rgb]{0.00,0.50,0.00}{##1}}}
\expandafter\def\csname PY@tok@nd\endcsname{\def\PY@tc##1{\textcolor[rgb]{0.67,0.13,1.00}{##1}}}
\expandafter\def\csname PY@tok@s\endcsname{\def\PY@tc##1{\textcolor[rgb]{0.73,0.13,0.13}{##1}}}
\expandafter\def\csname PY@tok@sd\endcsname{\let\PY@it=\textit\def\PY@tc##1{\textcolor[rgb]{0.73,0.13,0.13}{##1}}}
\expandafter\def\csname PY@tok@si\endcsname{\let\PY@bf=\textbf\def\PY@tc##1{\textcolor[rgb]{0.73,0.40,0.53}{##1}}}
\expandafter\def\csname PY@tok@se\endcsname{\let\PY@bf=\textbf\def\PY@tc##1{\textcolor[rgb]{0.73,0.40,0.13}{##1}}}
\expandafter\def\csname PY@tok@sr\endcsname{\def\PY@tc##1{\textcolor[rgb]{0.73,0.40,0.53}{##1}}}
\expandafter\def\csname PY@tok@ss\endcsname{\def\PY@tc##1{\textcolor[rgb]{0.10,0.09,0.49}{##1}}}
\expandafter\def\csname PY@tok@sx\endcsname{\def\PY@tc##1{\textcolor[rgb]{0.00,0.50,0.00}{##1}}}
\expandafter\def\csname PY@tok@m\endcsname{\def\PY@tc##1{\textcolor[rgb]{0.40,0.40,0.40}{##1}}}
\expandafter\def\csname PY@tok@gh\endcsname{\let\PY@bf=\textbf\def\PY@tc##1{\textcolor[rgb]{0.00,0.00,0.50}{##1}}}
\expandafter\def\csname PY@tok@gu\endcsname{\let\PY@bf=\textbf\def\PY@tc##1{\textcolor[rgb]{0.50,0.00,0.50}{##1}}}
\expandafter\def\csname PY@tok@gd\endcsname{\def\PY@tc##1{\textcolor[rgb]{0.63,0.00,0.00}{##1}}}
\expandafter\def\csname PY@tok@gi\endcsname{\def\PY@tc##1{\textcolor[rgb]{0.00,0.63,0.00}{##1}}}
\expandafter\def\csname PY@tok@gr\endcsname{\def\PY@tc##1{\textcolor[rgb]{1.00,0.00,0.00}{##1}}}
\expandafter\def\csname PY@tok@ge\endcsname{\let\PY@it=\textit}
\expandafter\def\csname PY@tok@gs\endcsname{\let\PY@bf=\textbf}
\expandafter\def\csname PY@tok@gp\endcsname{\let\PY@bf=\textbf\def\PY@tc##1{\textcolor[rgb]{0.00,0.00,0.50}{##1}}}
\expandafter\def\csname PY@tok@go\endcsname{\def\PY@tc##1{\textcolor[rgb]{0.53,0.53,0.53}{##1}}}
\expandafter\def\csname PY@tok@gt\endcsname{\def\PY@tc##1{\textcolor[rgb]{0.00,0.27,0.87}{##1}}}
\expandafter\def\csname PY@tok@err\endcsname{\def\PY@bc##1{\setlength{\fboxsep}{0pt}\fcolorbox[rgb]{1.00,0.00,0.00}{1,1,1}{\strut ##1}}}
\expandafter\def\csname PY@tok@kc\endcsname{\let\PY@bf=\textbf\def\PY@tc##1{\textcolor[rgb]{0.00,0.50,0.00}{##1}}}
\expandafter\def\csname PY@tok@kd\endcsname{\let\PY@bf=\textbf\def\PY@tc##1{\textcolor[rgb]{0.00,0.50,0.00}{##1}}}
\expandafter\def\csname PY@tok@kn\endcsname{\let\PY@bf=\textbf\def\PY@tc##1{\textcolor[rgb]{0.00,0.50,0.00}{##1}}}
\expandafter\def\csname PY@tok@kr\endcsname{\let\PY@bf=\textbf\def\PY@tc##1{\textcolor[rgb]{0.00,0.50,0.00}{##1}}}
\expandafter\def\csname PY@tok@bp\endcsname{\def\PY@tc##1{\textcolor[rgb]{0.00,0.50,0.00}{##1}}}
\expandafter\def\csname PY@tok@fm\endcsname{\def\PY@tc##1{\textcolor[rgb]{0.00,0.00,1.00}{##1}}}
\expandafter\def\csname PY@tok@vc\endcsname{\def\PY@tc##1{\textcolor[rgb]{0.10,0.09,0.49}{##1}}}
\expandafter\def\csname PY@tok@vg\endcsname{\def\PY@tc##1{\textcolor[rgb]{0.10,0.09,0.49}{##1}}}
\expandafter\def\csname PY@tok@vi\endcsname{\def\PY@tc##1{\textcolor[rgb]{0.10,0.09,0.49}{##1}}}
\expandafter\def\csname PY@tok@vm\endcsname{\def\PY@tc##1{\textcolor[rgb]{0.10,0.09,0.49}{##1}}}
\expandafter\def\csname PY@tok@sa\endcsname{\def\PY@tc##1{\textcolor[rgb]{0.73,0.13,0.13}{##1}}}
\expandafter\def\csname PY@tok@sb\endcsname{\def\PY@tc##1{\textcolor[rgb]{0.73,0.13,0.13}{##1}}}
\expandafter\def\csname PY@tok@sc\endcsname{\def\PY@tc##1{\textcolor[rgb]{0.73,0.13,0.13}{##1}}}
\expandafter\def\csname PY@tok@dl\endcsname{\def\PY@tc##1{\textcolor[rgb]{0.73,0.13,0.13}{##1}}}
\expandafter\def\csname PY@tok@s2\endcsname{\def\PY@tc##1{\textcolor[rgb]{0.73,0.13,0.13}{##1}}}
\expandafter\def\csname PY@tok@sh\endcsname{\def\PY@tc##1{\textcolor[rgb]{0.73,0.13,0.13}{##1}}}
\expandafter\def\csname PY@tok@s1\endcsname{\def\PY@tc##1{\textcolor[rgb]{0.73,0.13,0.13}{##1}}}
\expandafter\def\csname PY@tok@mb\endcsname{\def\PY@tc##1{\textcolor[rgb]{0.40,0.40,0.40}{##1}}}
\expandafter\def\csname PY@tok@mf\endcsname{\def\PY@tc##1{\textcolor[rgb]{0.40,0.40,0.40}{##1}}}
\expandafter\def\csname PY@tok@mh\endcsname{\def\PY@tc##1{\textcolor[rgb]{0.40,0.40,0.40}{##1}}}
\expandafter\def\csname PY@tok@mi\endcsname{\def\PY@tc##1{\textcolor[rgb]{0.40,0.40,0.40}{##1}}}
\expandafter\def\csname PY@tok@il\endcsname{\def\PY@tc##1{\textcolor[rgb]{0.40,0.40,0.40}{##1}}}
\expandafter\def\csname PY@tok@mo\endcsname{\def\PY@tc##1{\textcolor[rgb]{0.40,0.40,0.40}{##1}}}
\expandafter\def\csname PY@tok@ch\endcsname{\let\PY@it=\textit\def\PY@tc##1{\textcolor[rgb]{0.25,0.50,0.50}{##1}}}
\expandafter\def\csname PY@tok@cm\endcsname{\let\PY@it=\textit\def\PY@tc##1{\textcolor[rgb]{0.25,0.50,0.50}{##1}}}
\expandafter\def\csname PY@tok@cpf\endcsname{\let\PY@it=\textit\def\PY@tc##1{\textcolor[rgb]{0.25,0.50,0.50}{##1}}}
\expandafter\def\csname PY@tok@c1\endcsname{\let\PY@it=\textit\def\PY@tc##1{\textcolor[rgb]{0.25,0.50,0.50}{##1}}}
\expandafter\def\csname PY@tok@cs\endcsname{\let\PY@it=\textit\def\PY@tc##1{\textcolor[rgb]{0.25,0.50,0.50}{##1}}}

\def\PYZbs{\char`\\}
\def\PYZus{\char`\_}
\def\PYZob{\char`\{}
\def\PYZcb{\char`\}}
\def\PYZca{\char`\^}
\def\PYZam{\char`\&}
\def\PYZlt{\char`\<}
\def\PYZgt{\char`\>}
\def\PYZsh{\char`\#}
\def\PYZpc{\char`\%}
\def\PYZdl{\char`\$}
\def\PYZhy{\char`\-}
\def\PYZsq{\char`\'}
\def\PYZdq{\char`\"}
\def\PYZti{\char`\~}
% for compatibility with earlier versions
\def\PYZat{@}
\def\PYZlb{[}
\def\PYZrb{]}
\makeatother


    % Exact colors from NB
    \definecolor{incolor}{rgb}{0.0, 0.0, 0.5}
    \definecolor{outcolor}{rgb}{0.545, 0.0, 0.0}



    
    % Prevent overflowing lines due to hard-to-break entities
    \sloppy 
    % Setup hyperref package
    \hypersetup{
      breaklinks=true,  % so long urls are correctly broken across lines
      colorlinks=true,
      urlcolor=urlcolor,
      linkcolor=linkcolor,
      citecolor=citecolor,
      }
    % Slightly bigger margins than the latex defaults
    
    \geometry{verbose,tmargin=1in,bmargin=1in,lmargin=1in,rmargin=1in}
    
    

    \begin{document}
    
    
    \maketitle
    
    

    
    \section{TP Programmation en C : Introduction au traitement
d'images}\label{tp-programmation-en-c-introduction-au-traitement-dimages}

L'objectif de ce notebook est de vous permettre de faire un compte-rendu
de votre TP traitement d'image. Il s'agit une fois votre programme mis
au point de copier coller le code, et le résultats de vos tests.

\subsubsection{ATTENTION :}\label{attention}

\begin{itemize}
\item
  N'oubliez pas de travailler sous le répertoire sauvegardé sur le
  serveur dont le chemin est de la forme : /home/abc1234a/Documents
\item
  Lors de QCM vous devrez réutiliser votre code.Il faut donc pouvoir le
  retrouver.
\end{itemize}

\subsubsection{NOM ETUDIANT : GIBERT
ALEXIS}\label{nom-etudiant-gibert-alexis}

    \subsection{PARTIE 1- Editer, compiler et tester un
programme}\label{partie-1--editer-compiler-et-tester-un-programme}

    \section{PROGRAMME
transform\_image.c}\label{programme-transform_image.c}

\textbf{COPIER-COLLER ICI}

\textbf{(a) le résultat de la commande date}

\begin{verbatim}
gbl3344a@u3-201-11d:~/TPC/GIBERT_ALEXIS_TP_IMAGE$ date
jeu. 20 oct. 2022 09:43:59 CEST
\end{verbatim}

\textbf{(b) le code source correspondant}

\begin{verbatim}
#include <stdio.h>

void main(){
  int nbl,nbc,pxlc=0;

  scanf("%d",&nbl);
  scanf("%d",&nbc);

  for(int i=0;i<nbl;i++){
    for(int j=0;j<nbc;j++){
      scanf("%d",&pxlc);
      printf("%5d",((pxlc*4)/256));
    }
    printf("\n\r");
  }
}
\end{verbatim}

\textbf{(c) la commande de compilation et son résultat}

\begin{verbatim}
gbl3344a@u3-201-11d:~/TPC/GIBERT_ALEXIS_TP_IMAGE$ gcc transform_image.c -o transform_image.out
\end{verbatim}

\textbf{(d) le résultat des tests sur les fichiers de données}

\begin{verbatim}
gbl3344a@u3-201-11d:~/TPC/GIBERT_ALEXIS_TP_IMAGE$ ./transform_image.out < ./Data/image1.dat   
    0    0    0    0    0    0    0    0
    0    0    0    0    0    2    2    0
    0    0    0    0    2    2    0    0
    0    0    0    0    0    2    2    0
    0    1    0    0    0    2    0    0
    0    1    0    0    0    2    0    0
    0    1    1    0    2    2    2    0
    0    0    0    0    0    0    0    0
gbl3344a@u3-201-11d:~/TPC/GIBERT_ALEXIS_TP_IMAGE$ ./transform_image.out < ./Data/image2.dat
    0    0    0    0    0    0    0    0    0    0    0    0    0    0    0    0
    0    0    0    0    0    0    0    1    1    0    0    0    0    0    0    0
    0    0    0    0    0    0    0    1    1    0    0    0    0    0    0    0
    0    0    0    0    0    0    0    1    1    0    0    0    0    0    0    0
    0    0    0    0    0    0    0    1    1    0    0    0    0    0    0    0
    0    0    0    0    1    1    1    1    1    1    1    1    0    0    0    0
    0    0    0    0    1    1    1    1    1    1    1    1    0    0    0    0
    0    0    0    0    1    1    1    1    1    1    1    1    0    0    0    0
    0    0    0    0    1    1    1    1    1    1    1    1    0    0    0    0
    0    0    0    0    1    1    1    1    1    1    1    1    0    0    0    0!
    0    0    0    0    0    0    0    1    1    0    0    0    0    0    0    0
    0    0    0    0    0    0    0    1    1    0    0    0    0    0    0    0
    0    0    0    0    0    0    0    1    1    0    0    0    0    0    0    0
    0    0    0    0    0    0    0    1    1    0    0    0    0    0    0    0
    0    0    0    0    0    0    0    1    1    0    0    0    0    0    0    0
    0    0    0    0    0    0    0    0    0    0    0    0    0    0    0    0
\end{verbatim}

\textbf{(e) le contenu du répertoire TPC (fichiers et sous-répertoires)}

\begin{verbatim}
gbl3344a@u3-201-11d:~/TPC/GIBERT_ALEXIS_TP_IMAGE$ ls
Data                 NOTEBOOK_TP_IMAGE_GIBERT_ALEXIS.ipynb  transform_image.c
erreur_compil        Test                                   transform_image.out
MODULE_IMAGE_V0.tar  TP-image_2021.pdf
\end{verbatim}

    \subsection{PARTIE 2- Visualisation du résultat
(optionnel)!}\label{partie-2--visualisation-du-ruxe9sultat-optionnel}

\textbf{Avant la commande "imagesc(im2)" pour l'image 1}

\begin{figure}
\centering
\includegraphics{attachment:Capture\%20d\%E2\%80\%99\%C3\%A9cran_2022-10-27_08-18-02.png}
\caption{Capture\%20d\%E2\%80\%99\%C3\%A9cran\_2022-10-27\_08-18-02.png}
\end{figure}

\textbf{Après la commande "imagesc(im2)" pour l'image 1}

\begin{figure}
\centering
\includegraphics{attachment:Capture\%20d\%E2\%80\%99\%C3\%A9cran_2022-10-27_08-16-46.png}
\caption{Capture\%20d\%E2\%80\%99\%C3\%A9cran\_2022-10-27\_08-16-46.png}
\end{figure}

\textbf{Avant la commande "imagesc(im2)" pour l'image 2}

\begin{figure}
\centering
\includegraphics{attachment:Capture\%20d\%E2\%80\%99\%C3\%A9cran_2022-10-27_08-20-58.png}
\caption{Capture\%20d\%E2\%80\%99\%C3\%A9cran\_2022-10-27\_08-20-58.png}
\end{figure}

\textbf{Après la commande "imagesc(im2)" pour l'image 2}

\begin{figure}
\centering
\includegraphics{attachment:Capture\%20d\%E2\%80\%99\%C3\%A9cran_2022-10-27_08-21-38.png}
\caption{Capture\%20d\%E2\%80\%99\%C3\%A9cran\_2022-10-27\_08-21-38.png}
\end{figure}

    \subsection{PARTIE 3- Généraliser un traitement, traiter les cas
d'erreur}\label{partie-3--guxe9nuxe9raliser-un-traitement-traiter-les-cas-derreur}

\textbf{COPIER-COLLER ICI}

\subsubsection{(a) le résultat de la commande
date!}\label{a-le-ruxe9sultat-de-la-commande-date}

alexis@alexis-VirtualBox:/\$ date sam. 29 oct. 2022 19:31:25 CEST

\subsubsection{(b) le code source
correspondant}\label{b-le-code-source-correspondant}

\begin{verbatim}
#include <stdio.h>
#include <math.h>

void main(){
  int nbl,nbc,pxlc=0;
  int niveau;
  
  scanf("%d",&niveau);
  
  if(niveau<256){
    if((int)log2(niveau)==log2(niveau)){
      
      scanf("%d",&nbl);
      scanf("%d",&nbc);
      
      for(int i=0;i<nbl;i++){
        for(int j=0;j<nbc;j++){
          scanf("%d",&pxlc);
          printf("%5d",((pxlc*niveau)/256));
        }
        printf("\n\r");
      }
    }
    else printf("Erreur 1 : le niveau de gris est supérieur à 256");
  }
  else printf("Erreur 2 : le niveau de gris n’est pas une puissance de 2");
}
\end{verbatim}

\subsubsection{(c) la commande de compilation et son
résultat}\label{c-la-commande-de-compilation-et-son-ruxe9sultat}

\begin{verbatim}
alexis@alexis-VirtualBox:~/TPC/GIBERT_ALEXIS_TP_IMAGE$ gcc ./Programs/transform_image_gen.c -o ./Programs/transform_image_gen.out -lm
alexis@alexis-VirtualBox:~/TPC/GIBERT_ALEXIS_TP_IMAGE$ cat ./Programs/niveau ./Data/image1.dat | ./Programs/transform_image_gen.out > ./Data/image1_gen.dat
alexis@alexis-VirtualBox:~/TPC/GIBERT_ALEXIS_TP_IMAGE$ cat < ./Data/image1.dat
    0    0    0    0    0    0    0    0
    0    2    2    0    0    9    9    0
    0    2    2    0    9    9    0    0
    0    0    0    0    0    9    9    0
    0    5    0    0    0   10    0    0
    0    5    0    0    0   10    0    0
    0    5    5    0   10   10   10    0
    0    0    0    0    0    0    0    0
\end{verbatim}

\subsubsection{(d) le résultat des tests -- question
15}\label{d-le-ruxe9sultat-des-tests-question-15}

\textbf{test\_0.res}

\begin{verbatim}
alexis@alexis-VirtualBox:~/TPC/GIBERT_ALEXIS_TP_IMAGE$ cat > ./Programs/niveau
0
alexis@alexis-VirtualBox:~/TPC/GIBERT_ALEXIS_TP_IMAGE$ gcc ./Programs/transform_image_gen.c -o ./Programs/transform_image_gen.out -lm
alexis@alexis-VirtualBox:~/TPC/GIBERT_ALEXIS_TP_IMAGE$ cat ./Programs/niveau ./Data/image1.dat | ./Programs/transform_image_gen.out > ./Test/test_0.res
alexis@alexis-VirtualBox:~/TPC/GIBERT_ALEXIS_TP_IMAGE$ cat < ./Test/test_0.res 
Erreur 2 : le niveau de gris n’est pas une puissance de 2
\end{verbatim}

\textbf{test\_2.res}

\begin{verbatim}
alexis@alexis-VirtualBox:~/TPC/GIBERT_ALEXIS_TP_IMAGE$ cat > ./Programs/niveau
2
alexis@alexis-VirtualBox:~/TPC/GIBERT_ALEXIS_TP_IMAGE$ gcc ./Programs/transform_image_gen.c -o ./Programs/transform_image_gen.out -lm
alexis@alexis-VirtualBox:~/TPC/GIBERT_ALEXIS_TP_IMAGE$ cat ./Programs/niveau ./Data/image1.dat | ./Programs/transform_image_gen.out > ./Test/test_2.res
alexis@alexis-VirtualBox:~/TPC/GIBERT_ALEXIS_TP_IMAGE$ cat < ./Test/test_2.res 
    0    0    0    0    0    0    0    0
    0    0    0    0    0    1    1    0
    0    0    0    0    1    1    0    0
    0    0    0    0    0    1    1    0
    0    0    0    0    0    1    0    0
    0    0    0    0    0    1    0    0
    0    0    0    0    1    1    1    0
    0    0    0    0    0    0    0    0
\end{verbatim}

\textbf{test\_4.res}

\begin{verbatim}
alexis@alexis-VirtualBox:~/TPC/GIBERT_ALEXIS_TP_IMAGE$ cat > ./Programs/niveau
4
alexis@alexis-VirtualBox:~/TPC/GIBERT_ALEXIS_TP_IMAGE$ gcc ./Programs/transform_image_gen.c -o ./Programs/transform_image_gen.out -lm
alexis@alexis-VirtualBox:~/TPC/GIBERT_ALEXIS_TP_IMAGE$ cat ./Programs/niveau ./Data/image1.dat | ./Programs/transform_image_gen.out > ./Test/test_4.res
alexis@alexis-VirtualBox:~/TPC/GIBERT_ALEXIS_TP_IMAGE$ cat < ./Test/test_4.res 
    0    0    0    0    0    0    0    0
    0    0    0    0    0    2    2    0
    0    0    0    0    2    2    0    0
    0    0    0    0    0    2    2    0
    0    1    0    0    0    2    0    0
    0    1    0    0    0    2    0    0
    0    1    1    0    2    2    2    0
    0    0    0    0    0    0    0    0
\end{verbatim}

\textbf{test\_15.res}

\begin{verbatim}
alexis@alexis-VirtualBox:~/TPC/GIBERT_ALEXIS_TP_IMAGE$ cat > ./Programs/niveau
15
alexis@alexis-VirtualBox:~/TPC/GIBERT_ALEXIS_TP_IMAGE$ gcc ./Programs/transform_image_gen.c -o ./Programs/transform_image_gen.out -lm
alexis@alexis-VirtualBox:~/TPC/GIBERT_ALEXIS_TP_IMAGE$ cat ./Programs/niveau ./Data/image1.dat | ./Programs/transform_image_gen.out > ./Test/test_15.res
alexis@alexis-VirtualBox:~/TPC/GIBERT_ALEXIS_TP_IMAGE$ cat < ./Test/test_15.res 
Erreur 2 : le niveau de gris n’est pas une puissance de 2
\end{verbatim}

\textbf{test\_32.res}

\begin{verbatim}
alexis@alexis-VirtualBox:~/TPC/GIBERT_ALEXIS_TP_IMAGE$ cat > ./Programs/niveau
32
alexis@alexis-VirtualBox:~/TPC/GIBERT_ALEXIS_TP_IMAGE$ gcc ./Programs/transform_image_gen.c -o ./Programs/transform_image_gen.out -lm
alexis@alexis-VirtualBox:~/TPC/GIBERT_ALEXIS_TP_IMAGE$ cat ./Programs/niveau ./Data/image1.dat | ./Programs/transform_image_gen.out > ./Test/test_32.res
alexis@alexis-VirtualBox:~/TPC/GIBERT_ALEXIS_TP_IMAGE$ cat < ./Test/test_32.res 
    0    0    0    0    0    0    0    0
    0    4    4    0    0   18   18    0
    0    4    4    0   18   18    0    0
    0    0    0    0    0   18   18    0
    0   11    0    0    0   21    0    0
    0   11    0    0    0   21    0    0
    0   11   11    0   21   21   21    0
    0    0    0    0    0    0    0    0
\end{verbatim}

\textbf{test\_260.res}

\begin{verbatim}
alexis@alexis-VirtualBox:~/TPC/GIBERT_ALEXIS_TP_IMAGE$ cat > ./Programs/niveau
260
alexis@alexis-VirtualBox:~/TPC/GIBERT_ALEXIS_TP_IMAGE$ gcc ./Programs/transform_image_gen.c -o ./Programs/transform_image_gen.out -lm
alexis@alexis-VirtualBox:~/TPC/GIBERT_ALEXIS_TP_IMAGE$ cat ./Programs/niveau ./Data/image1.dat | ./Programs/transform_image_gen.out > ./Test/test_260.res
alexis@alexis-VirtualBox:~/TPC/GIBERT_ALEXIS_TP_IMAGE$ cat < ./Test/test_260.res
Erreur 1 : le niveau de gris est supérieur à 256
\end{verbatim}

\subsubsection{(e) le contenu du répertoire TPC (fichiers et
sous-répertoires)}\label{e-le-contenu-du-ruxe9pertoire-tpc-fichiers-et-sous-ruxe9pertoires}

\begin{verbatim}
alexis@alexis-VirtualBox:~/TPC$ ls -R
.:
GIBERT_ALEXIS_TP_IMAGE  GIBERT_ALEXIS_TP_IMAGE.tar

./GIBERT_ALEXIS_TP_IMAGE:
Data                                   Pictures  TP-image_2021.pdf
MODULE_IMAGE_V0.tar                    Programs
NOTEBOOK_TP_IMAGE_GIBERT_ALEXIS.ipynb  Test

./GIBERT_ALEXIS_TP_IMAGE/Data:
image1_4N.dat  image1_gen.dat         image2_4N.dat  image2_without_lc.dat
image1.dat     image1_without_lc.dat  image2.dat     mon_image_seule.dat

./GIBERT_ALEXIS_TP_IMAGE/Pictures:
'Capture d’écran_2022-10-27_08-16-46.png'
'Capture d’écran_2022-10-27_08-18-02.png'
'Capture d’écran_2022-10-27_08-20-58.png'
'Capture d’écran_2022-10-27_08-21-38.png'

./GIBERT_ALEXIS_TP_IMAGE/Programs:
erreur_compil  transform_image.c      transform_image_gen.out
niveau         transform_image_gen.c  transform_image.out

./GIBERT_ALEXIS_TP_IMAGE/Test:
test_0.res  test_15.res  test_260.res  test_2.res  test_32.res  test_4.res
\end{verbatim}

    \subsection{PARTIE 4- Notion de module et de compilation
séparée}\label{partie-4--notion-de-module-et-de-compilation-suxe9paruxe9e}

\textbf{COPIER-COLLER ICI}

\subsubsection{(a) le résultat de la commande
date}\label{a-le-ruxe9sultat-de-la-commande-date}

\begin{verbatim}
alexis@alexis-VirtualBox:~/TPC$ date
sam. 29 oct. 2022 20:08:33 CEST
\end{verbatim}

\subsubsection{(b) le contenu du répertoire TPC (fichiers et
sous-répertoires)}\label{b-le-contenu-du-ruxe9pertoire-tpc-fichiers-et-sous-ruxe9pertoires}

\begin{verbatim}
alexis@alexis-VirtualBox:~/TPC$ ls -R
.:
GIBERT_ALEXIS_MODULE_IMAGE  GIBERT_ALEXIS_TP_IMAGE.tar
GIBERT_ALEXIS_TP_IMAGE      TP-image_2021.pdf

./GIBERT_ALEXIS_MODULE_IMAGE:
Data  MODULE_IMAGE  MODULE_IMAGE_V0.tar  Test

./GIBERT_ALEXIS_MODULE_IMAGE/Data:

./GIBERT_ALEXIS_MODULE_IMAGE/MODULE_IMAGE:
ferrane_module_image.c  ferrane_module_image.h  main.c  Makefile

./GIBERT_ALEXIS_MODULE_IMAGE/Test:

./GIBERT_ALEXIS_TP_IMAGE:
Data  NOTEBOOK_TP_IMAGE_GIBERT_ALEXIS.ipynb  Pictures  Programs  Test

./GIBERT_ALEXIS_TP_IMAGE/Data:
image1_4N.dat  image1_gen.dat         image2_4N.dat  image2_without_lc.dat
image1.dat     image1_without_lc.dat  image2.dat     mon_image_seule.dat

./GIBERT_ALEXIS_TP_IMAGE/Pictures:
'Capture d’écran_2022-10-27_08-16-46.png'
'Capture d’écran_2022-10-27_08-18-02.png'
'Capture d’écran_2022-10-27_08-20-58.png'
'Capture d’écran_2022-10-27_08-21-38.png'

./GIBERT_ALEXIS_TP_IMAGE/Programs:
erreur_compil  transform_image.c      transform_image_gen.out
niveau         transform_image_gen.c  transform_image.out

./GIBERT_ALEXIS_TP_IMAGE/Test:
test_0.res  test_15.res  test_260.res  test_2.res  test_32.res  test_4.res
\end{verbatim}

\textbf{Question 20 :} - Le fichier "ferrane\_module\_image.h" contient
la déclaration de la fonction "void test\_prog(void);" - Le fihier
"ferrane\_module\_image.c" contient la définition de la fonction
déclarée dans le fichier "ferrane\_module\_image.h" soit le contenu de
la fonction "void test\_prog(void);". - Le fichier "main.c" utilise la
fonction "void test\_prog(void);" a travers le fichier
"ferrane\_module\_image.h" grâce la déclaration "\#include
"ferrane\_module\_image.h"".

\subsubsection{(c) le contenu du Makefile de la question
21}\label{c-le-contenu-du-makefile-de-la-question-21}

\textbf{Question 21 :}

Contenu du makefile :

\begin{verbatim}
mon_module_image.out: mon_module_image.o main.o
    gcc -o mon_module_image.out mon_module_image.o main.o
mon_module_image.o: ferrane_module_image.c 
    gcc -c ferrane_module_image.c -o mon_module_image.o  -W -Wall -ansi -pedantic
main.o: main.c ferrane_module_image.h
    gcc -o main.o -c main.c -W -Wall -ansi -pedantic
clean:
    rm -rf *.o
\end{verbatim}

\begin{itemize}
\tightlist
\item
  \textbf{Le "Makefile" contient un script qui suit le procesus suivant
  :}

  \begin{itemize}
  \tightlist
  \item
    Les dépendances "main.c" et "ferrane\_module\_image.h" sont rangées
    dans la cible intermédiaire "main.o" puis la compile
  \item
    La dépendance "ferrane\_module\_image.c" est rangé dans la cible
    intermédiaire "mon\_module\_image.o" puis la compile
  \item
    Les dépendances "mon\_module\_image.o" "main.o" sont rangées dans la
    cible finale "mon\_module\_image.out" puis la compile
  \item
    (Par la suite les fichiers intermédiaires .o pourront être supprimés
    en utilisant la commande "make clean")
  \end{itemize}
\item
  \textbf{Les options utilisées :}

  \begin{itemize}
  \item
    \texttt{-W},\texttt{-Wall} et \texttt{-pedantic} sont des options
    relatives aux "warning". Celles-ci activent tous les "warning"
    relatif au fichier compilé et conseille l'utilisateur afin de parrer
    une éventuelle erreur.
  \item
    \texttt{-ansi} (nécessite \texttt{-pedantic}) spécifie la norme à
    laquelle le code doit se conformer. Actuellement, CPP connaît les
    normes C et C++~; d'autres pourraient être ajoutés à l'avenir.
    L'option \texttt{-ansi} est équivalente à \texttt{-std=c89}, soit la
    norme "c89".
  \end{itemize}
\end{itemize}

\subsubsection{(d) les réponses aux questions 22, 23,
24}\label{d-les-ruxe9ponses-aux-questions-22-23-24}

\textbf{Question 22 :}

\begin{verbatim}
alexis@alexis-VirtualBox:~/TPC/GIBERT_ALEXIS_MODULE_IMAGE/MODULE_IMAGE$ make mon_module_image.out
gcc -c ferrane_module_image.c -o mon_module_image.o  -W -Wall -ansi -pedantic
gcc -o main.o -c main.c -W -Wall -ansi -pedantic
gcc -o mon_module_image.out mon_module_image.o main.o
alexis@alexis-VirtualBox:~/TPC/GIBERT_ALEXIS_MODULE_IMAGE/MODULE_IMAGE$ ls
ferrane_module_image.c  main.c  Makefile            mon_module_image.out
ferrane_module_image.h  main.o  mon_module_image.o
\end{verbatim}

On observe que le script a executé les différentes commandes présentées
précédemment et a généré : - La dépendence intermédiaire "main.o" - La
dépendence intermédiaire "mon\_module\_image.o" - La dépendance finale /
exécutable "mon\_module\_image.out"

\textbf{Question 23 :}

\begin{verbatim}
alexis@alexis-VirtualBox:~/TPC/GIBERT_ALEXIS_MODULE_IMAGE/MODULE_IMAGE$ ./mon_module_image.out
DEBUT DE TEST MODULE 
TEST DU MODULE : FERRANE_MODULE_IMAGE
DEBUT DE TEST MODULE 
\end{verbatim}

On remarque que la fonction du fichier "ferrane\_module\_image.c" a bien
été exécuté au travers du fichier "main.c" et du fichier
"ferrane\_module\_image.h".

\textbf{Question 24 :}

\begin{verbatim}
alexis@alexis-VirtualBox:~/TPC/GIBERT_ALEXIS_MODULE_IMAGE/MODULE_IMAGE$ make clean
rm -rf *.o
alexis@alexis-VirtualBox:~/TPC/GIBERT_ALEXIS_MODULE_IMAGE/MODULE_IMAGE$ ls
ferrane_module_image.c  main.c    mon_module_image.out
ferrane_module_image.h  Makefile
\end{verbatim}

On remarque que les fichiers .o ont bien été supprimé.

\textbf{Question 25 :}

Par la suite, pour traiter la suite des questions, on renomme le
répertoire "MODULE\_IMAGE" en "MODULE\_IMAGE\_V0" puis on le copie en
prenant soin de renommer la copie : "MODULE\_IMAGE\_V1"

\begin{verbatim}
alexis@alexis-VirtualBox:~/TPC/GIBERT_ALEXIS_MODULE_IMAGE$ mv MODULE_IMAGE MODULE_IMAGE_V0
alexis@alexis-VirtualBox:~/TPC/GIBERT_ALEXIS_MODULE_IMAGE$ ls
Data  MODULE_IMAGE_V0  MODULE_IMAGE_V0.tar  Test
alexis@alexis-VirtualBox:~/TPC/GIBERT_ALEXIS_MODULE_IMAGE$ cp -r MODULE_IMAGE_V0 MODULE_IMAGE_V1
alexis@alexis-VirtualBox:~/TPC/GIBERT_ALEXIS_MODULE_IMAGE$ ls
Data  MODULE_IMAGE_V0  MODULE_IMAGE_V0.tar  MODULE_IMAGE_V1  Test
\end{verbatim}

\textbf{Question 26 :}

Ainsi, dans le répertoire "MODULE\_IMAGE\_V1" afin de ce retrouver avec
l'arborescence - main.c - Makefile - votre\_nom\_module\_image.c -
votre\_nom\_module\_image.h

on utilisera les commandes suivantes :

\begin{verbatim}
alexis@alexis-VirtualBox:~/TPC/GIBERT_ALEXIS_MODULE_IMAGE$ cd MODULE_IMAGE_V1
alexis@alexis-VirtualBox:~/TPC/GIBERT_ALEXIS_MODULE_IMAGE/MODULE_IMAGE_V1$ ls
ferrane_module_image.c  main.c    mon_module_image.out
ferrane_module_image.h  Makefile
alexis@alexis-VirtualBox:~/TPC/GIBERT_ALEXIS_MODULE_IMAGE/MODULE_IMAGE_V1$ mv ferrane_module_image.c gibert_module_image.c
alexis@alexis-VirtualBox:~/TPC/GIBERT_ALEXIS_MODULE_IMAGE/MODULE_IMAGE_V1$ mv ferrane_module_image.h gibert_module_image.h
alexis@alexis-VirtualBox:~/TPC/GIBERT_ALEXIS_MODULE_IMAGE/MODULE_IMAGE_V1$ rm mon_module_image.out
alexis@alexis-VirtualBox:~/TPC/GIBERT_ALEXIS_MODULE_IMAGE/MODULE_IMAGE_V1$ ls
gibert_module_image.c  gibert_module_image.h  main.c  Makefile
\end{verbatim}

\textbf{Question 27 :}

Si on tente de compiler le Makefile on remarque que :

\begin{verbatim}
alexis@alexis-VirtualBox:~/TPC/GIBERT_ALEXIS_MODULE_IMAGE/MODULE_IMAGE_V1$ make mon_module_image.out
make: *** No rule to make target 'ferrane_module_image.c', needed by 'mon_module_image.o'. Stop.
\end{verbatim}

Puisque le fait d'avoir renommé les fichier .c et .h a pour effet de
rompre les liens interne du Makefile.

\begin{verbatim}
alexis@alexis-VirtualBox:~/TPC/GIBERT_ALEXIS_MODULE_IMAGE/MODULE_IMAGE_V1$ cat Makefile 
mon_module_image.out: mon_module_image.o main.o
    gcc -o mon_module_image.out mon_module_image.o main.o

mon_module_image.o: ferrane_module_image.c 
    gcc -c ferrane_module_image.c -o mon_module_image.o  -W -Wall -ansi -pedantic

main.o: main.c ferrane_module_image.h
    gcc -o main.o -c main.c -W -Wall -ansi -pedantic
    
clean:
    rm -rf *.o
\end{verbatim}

Il va donc falloir re-créer les liens en modifiant le script du
Makefile. Pour cela j'ai installé l'application nano qui va me permettre
de modifier le fichier Makefile tout en restant dans le shell linux.

\begin{verbatim}
alexis@alexis-VirtualBox:~/TPC/GIBERT_ALEXIS_MODULE_IMAGE/MODULE_IMAGE_V1$ sudo apt-get install nano
[sudo] password for alexis: 
Reading package lists... Done
Building dependency tree       
Reading state information... Done
nano is already the newest version (4.8-1ubuntu1).
nano set to manually installed.
The following packages were automatically installed and are no longer required:
  linux-headers-5.13.0-28-generic linux-hwe-5.13-headers-5.13.0-28
  linux-image-5.13.0-28-generic linux-modules-5.13.0-28-generic
  linux-modules-extra-5.13.0-28-generic pandoc-data
Use 'sudo apt autoremove' to remove them.
0 to upgrade, 0 to newly install, 0 to remove and 306 not to upgrade.
\end{verbatim}

Ainsi à l'aide de la commande "nano Makefile" je peut alors éditer le
fichier Makefile.

\begin{longtable}[]{@{}cc@{}}
\toprule
\begin{minipage}[b]{0.47\columnwidth}\centering\strut
Le fichier Makefile dans nano (avant modification)\strut
\end{minipage} & \begin{minipage}[b]{0.47\columnwidth}\centering\strut
Le fichier Makefile dans nano (après modification)\strut
\end{minipage}\tabularnewline
\midrule
\endhead
\begin{minipage}[t]{0.47\columnwidth}\centering\strut
\includegraphics{attachment:TP_IMAGE_NANO1.jpg}\strut
\end{minipage} & \begin{minipage}[t]{0.47\columnwidth}\centering\strut
\includegraphics{attachment:TP_IMAGE_NANO2.jpg}\strut
\end{minipage}\tabularnewline
\bottomrule
\end{longtable}

\textbf{Nouvelle erreur :}

\begin{verbatim}
alexis@alexis-VirtualBox:~/TPC/GIBERT_ALEXIS_MODULE_IMAGE/MODULE_IMAGE_V1$ make mon_module_image.out
gcc -c gibert_module_image.c -o mon_module_image.o  -W -Wall -ansi -pedantic
gibert_module_image.c:3:10: fatal error: ferrane_module_image.h: No such file or directory
    3 | #include "ferrane_module_image.h"
      |          ^~~~~~~~~~~~~~~~~~~~~~~~
compilation terminated.
make: *** [Makefile:5: mon_module_image.o] Error 1
\end{verbatim}

Il va donc falloir modifier l'include du fichier
"gibert\_module\_image.c" et du fichier "main.c" :

\begin{verbatim}
alexis@alexis-VirtualBox:~/TPC/GIBERT_ALEXIS_MODULE_IMAGE/MODULE_IMAGE_V1$ nano gibert_module_image.c
alexis@alexis-VirtualBox:~/TPC/GIBERT_ALEXIS_MODULE_IMAGE/MODULE_IMAGE_V1$ nano main.c
\end{verbatim}

\begin{longtable}[]{@{}cc@{}}
\toprule
\begin{minipage}[b]{0.47\columnwidth}\centering\strut
Le fichier .c dans nano (avant modification)\strut
\end{minipage} & \begin{minipage}[b]{0.47\columnwidth}\centering\strut
Le fichier .c dans nano (après modification)\strut
\end{minipage}\tabularnewline
\midrule
\endhead
\begin{minipage}[t]{0.47\columnwidth}\centering\strut
\includegraphics{attachment:TP_IMAGE_NANO3.jpg}\strut
\end{minipage} & \begin{minipage}[t]{0.47\columnwidth}\centering\strut
\includegraphics{attachment:TP_IMAGE_NANO4.jpg}\strut
\end{minipage}\tabularnewline
\begin{minipage}[t]{0.47\columnwidth}\centering\strut
\includegraphics{attachment:TP_IMAGE_NANO5.jpg}\strut
\end{minipage} & \begin{minipage}[t]{0.47\columnwidth}\centering\strut
\includegraphics{attachment:TP_IMAGE_NANO6.jpg}\strut
\end{minipage}\tabularnewline
\bottomrule
\end{longtable}

Ainsi maintenant tout ce compile correctement :

\begin{verbatim}
alexis@alexis-VirtualBox:~/TPC/GIBERT_ALEXIS_MODULE_IMAGE/MODULE_IMAGE_V1$ make mon_module_image.out
gcc -o main.o -c main.c -W -Wall -ansi -pedantic
gcc -o mon_module_image.out mon_module_image.o main.o
alexis@alexis-VirtualBox:~/TPC/GIBERT_ALEXIS_MODULE_IMAGE/MODULE_IMAGE_V1$ ls
gibert_module_image.c  main.c  Makefile            mon_module_image.out
gibert_module_image.h  main.o  mon_module_image.o
\end{verbatim}

On vérifie ensuite que l'arborescence correspond a celle souhaitée :

\begin{verbatim}
alexis@alexis-VirtualBox:~/TPC$ ls -R
.:
GIBERT_ALEXIS_MODULE_IMAGE  TP_IMAGE_NANO1.jpg  TP_IMAGE_NANO5.jpg
GIBERT_ALEXIS_TP_IMAGE      TP_IMAGE_NANO2.jpg  TP_IMAGE_NANO6.jpg
GIBERT_ALEXIS_TP_IMAGE.tar  TP_IMAGE_NANO3.jpg
TP-image_2021.pdf           TP_IMAGE_NANO4.jpg

./GIBERT_ALEXIS_MODULE_IMAGE:
Data  MODULE_IMAGE_V0  MODULE_IMAGE_V0.tar  MODULE_IMAGE_V1  Test

...

./GIBERT_ALEXIS_TP_IMAGE:
Data  NOTEBOOK_TP_IMAGE_GIBERT_ALEXIS.ipynb  Pictures  Programs  Test

...
\end{verbatim}

On se retrouve avec un arborescence quasi-similaire mis a part que j'ai
choisi de ranger les programmes et les images dans des repertoires. Pour
être en accord avec la consigne je vais donc : - Déplacer le notebook
dans le répertoire racine du TP soit le répertoire "TPC"

\begin{verbatim}
alexis@alexis-VirtualBox:~/TPC$ mv ./GIBERT_ALEXIS_TP_IMAGE/NOTEBOOK_TP_IMAGE_GIBERT_ALEXIS.ipynb ~/TPC
alexis@alexis-VirtualBox:~/TPC$ ls
GIBERT_ALEXIS_MODULE_IMAGE             TP-image_2021.pdf   TP_IMAGE_NANO4.jpg
GIBERT_ALEXIS_TP_IMAGE                 TP_IMAGE_NANO1.jpg  TP_IMAGE_NANO5.jpg
GIBERT_ALEXIS_TP_IMAGE.tar             TP_IMAGE_NANO2.jpg  TP_IMAGE_NANO6.jpg
NOTEBOOK_TP_IMAGE_GIBERT_ALEXIS.ipynb  TP_IMAGE_NANO3.jpg
\end{verbatim}

\begin{itemize}
\tightlist
\item
  Remonter d'un étage les fichiers du répertoire "Programs" et avant de
  le supprimer
\end{itemize}

\begin{verbatim}
alexis@alexis-VirtualBox:~/TPC$ cd GIBERT_ALEXIS_TP_IMAGE/
alexis@alexis-VirtualBox:~/TPC/GIBERT_ALEXIS_TP_IMAGE$ ls
Data  Pictures  Programs  Test
alexis@alexis-VirtualBox:~/TPC/GIBERT_ALEXIS_TP_IMAGE$ mv ./Programs/** ~/TPC/GIBERT_ALEXIS_TP_IMAGE
alexis@alexis-VirtualBox:~/TPC/GIBERT_ALEXIS_TP_IMAGE$ ls
Data           Pictures  transform_image.c        transform_image.out
erreur_compil  Programs  transform_image_gen.c
niveau         Test      transform_image_gen.out
alexis@alexis-VirtualBox:~/TPC/GIBERT_ALEXIS_TP_IMAGE$ rmdir Programs/
\end{verbatim}

\begin{itemize}
\tightlist
\item
  Regrouper l'ensemble des photos au sein d'un seul et même répertoire
  Pitures situé dans le répertoire TPC
\end{itemize}

\begin{verbatim}
alexis@alexis-VirtualBox:~/TPC/GIBERT_ALEXIS_TP_IMAGE$ mv ./Pictures ~/TPC
alexis@alexis-VirtualBox:~/TPC/GIBERT_ALEXIS_TP_IMAGE$ ls
Data           niveau  transform_image.c      transform_image_gen.out
erreur_compil  Test    transform_image_gen.c  transform_image.out
alexis@alexis-VirtualBox:~/TPC/GIBERT_ALEXIS_TP_IMAGE$ cd ..
alexis@alexis-VirtualBox:~/TPC$ ls
GIBERT_ALEXIS_MODULE_IMAGE             Pictures            TP_IMAGE_NANO3.jpg
GIBERT_ALEXIS_TP_IMAGE                 TP-image_2021.pdf   TP_IMAGE_NANO4.jpg
GIBERT_ALEXIS_TP_IMAGE.tar             TP_IMAGE_NANO1.jpg  TP_IMAGE_NANO5.jpg
NOTEBOOK_TP_IMAGE_GIBERT_ALEXIS.ipynb  TP_IMAGE_NANO2.jpg  TP_IMAGE_NANO6.jpg
alexis@alexis-VirtualBox:~/TPC$ mv ./*.jpg ./Pictures/
alexis@alexis-VirtualBox:~/TPC$ ls
GIBERT_ALEXIS_MODULE_IMAGE  NOTEBOOK_TP_IMAGE_GIBERT_ALEXIS.ipynb
GIBERT_ALEXIS_TP_IMAGE      Pictures
\end{verbatim}

Ainsi on a bien l'arborescence souhaitée

\begin{verbatim}
alexis@alexis-VirtualBox:~/TPC$ ls -R
.:
GIBERT_ALEXIS_MODULE_IMAGE  NOTEBOOK_TP_IMAGE_GIBERT_ALEXIS.ipynb
GIBERT_ALEXIS_TP_IMAGE      Pictures
GIBERT_ALEXIS_TP_IMAGE.tar  TP-image_2021.pdf

./GIBERT_ALEXIS_MODULE_IMAGE:
Data  MODULE_IMAGE_V0  MODULE_IMAGE_V0.tar  MODULE_IMAGE_V1  Test

...

./GIBERT_ALEXIS_TP_IMAGE:
Data           niveau  transform_image.c      transform_image_gen.out
erreur_compil  Test    transform_image_gen.c  transform_image.out
...
\end{verbatim}

\subsubsection{~(e) le résultat de la commande
history}\label{e-le-ruxe9sultat-de-la-commande-history}

alexis@alexis-VirtualBox:\textasciitilde{}/TPC\$ history

\begin{verbatim}
...
  211  cd TPC/Programs/
  212  gcc transform_image_gen.c -o transform_image_gen.out
  213  gcc transform_image_gen.c -o transform_image_gen.out -lm
  214  cat niveau ./Data/image1.dat | ./Program/transform_image.out > image_gen.dat
  215  cd ..
  216  cat niveau ./Data/image1.dat | ./Program/transform_image.out > image_gen.dat
  217  cat niveau ./Data/image1.dat | ./Programs/transform_image.out > image_gen.dat
  218  cat ./Programs/niveau ./Data/image1.dat | ./Programs/transform_image.out > image_gen.dat
  219  ls -l
  220  cd Programs
  221  ls -l
  222  chmod transform_image_gen.out u+x
  223  chmod transform_image_gen.out -u+x
  224  ls -l
  225  chmod transform_image_gen.out -u+rwx
  226  ls -l
  227  cd ..
  228  cat ./Programs/niveau ./Data/image1.dat | ./Programs/transform_image.out > image_gen.dat
  229  cat ./Programs/niveau ./Data/image1.dat | ./Programs/transform_image_gen.out > image_gen.dat
  230  cat ./Programs/niveau ./Data/image1_without_lc.dat | ./Programs/transform_image_gen.out > image_gen.dat
  231  gcc transform_image_gen.c -o transform_image_gen.out -lm
  232  gcc ./Programs/transform_image_gen.c -o ./Programs/transform_image_gen.out -lm
  233  cat ./Programs/niveau ./Data/image1_without_lc.dat | ./Programs/transform_image_gen.out > image_gen.dat
  234  cat ./Programs/niveau ./Data/image1.dat | ./Programs/transform_image_gen.out > image_gen.dat
  235  cat ./Programs/niveau ./Data/image1.dat | ./Programs/transform_image_gen.out > ./Data/image_gen.dat
  236  cat ./Programs/niveau ./Data/image1.dat | ./Programs/transform_image_gen.out > ./Data/image1_gen.dat
  237  gcc ./Programs/transform_image_gen.c -o ./Programs/transform_image_gen.out -lm
  238  cat ./Programs/niveau ./Data/image1.dat | ./Programs/transform_image_gen.out > ./Data/image1_gen.dat
  239  cat > ./Programs/niveau
  240  cat ./Programs/niveau ./Data/image1.dat | ./Programs/transform_image_gen.out > ./Test/test_0.res
  241  cat < ./Test/test_0.res 
  242  cat > ./Programs/niveau
  243  cat ./Programs/niveau ./Data/image1.dat | ./Programs/transform_image_gen.out > ./Test/test_4.res
  244  cat < ./Test/test_4.res 
  245  cat > ./Programs/niveau
  246  cat ./Programs/niveau ./Data/image1.dat | ./Programs/transform_image_gen.out > ./Test/test_2.res
  247  cat < ./Test/test_2.res 
  248  cat > ./Programs/niveau
  249  cat ./Programs/niveau ./Data/image1.dat | ./Programs/transform_image_gen.out > ./Test/test_15.res
  250  cat < ./Test/test_15.res 
  251  cat > ./Programs/niveau
  252  cat ./Programs/niveau ./Data/image1.dat | ./Programs/transform_image_gen.out > ./Test/test_32.res
  253  cat < ./Test/test_32.res 
  254  cat > ./Programs/niveau
  255  cat ./Programs/niveau ./Data/image1.dat | ./Programs/transform_image_gen.out > ./Test/test_260.res
  256  cat < ./Test/test_260.res 
  257  tar -cvf GIBERT_ALEXIS_TP_IMAGE.tar GIBERT_Alexis_TP_IMAGE
  258  tar -cvf GIBERT_ALEXIS_TP_IMAGE.tar GIBERT_ALEXIS_TP_IMAGE
  259  date
  260  cd..
  261  cd ..
  262  sudo apt install anaconda
  263  sudo apt-get install anaconda
  264  cd Downloads/
  265  sudo apt install anaconda
  266  sudo apt-get install anaconda
  267  sudo install Anaconda3-2022.10-Linux-x86_64.sh 
  268  sudo apt install Anaconda3-2022.10-Linux-x86_64.sh 
  269*  Anaconda3-2022.10-Linux-x86_64.sh 
  270  cat > Anaconda3-2022.10-Linux-x86_64.sh 
  271  cat Anaconda3-2022.10-Linux-x86_64.sh 
  272  cd /tmp
  273  curl -O https://repo.anaconda.com/archive/Anaconda3-5.2.0-Linux-x86_64.sh
  274  sudo apt install curl
  275  curl -O https://repo.anaconda.com/archive/Anaconda3-5.2.0-Linux-x86_64.sh
  276  sha256sum Anaconda3-5.2.0-Linux-x86_64.sh
  277  bash Anaconda3-5.2.0-Linux-x86_64.sh
  278  jupyter notebook
  279  sudo apt install jupyter
  280  jupyter notebook
  281  cd ..
  282  date
  283  cd ~
  284  ls
  285  cd TPC
  286  ls
  287  cd GIBERT_ALEXIS_TP_IMAGE/
  288  cat < ./Data/image1_gen.dat 
  289  cat < ./Test/test_0.res 
  290  cd ..
  291  ls
  292  ls -a
  293  man ls
  294  ls -l
  295  tree
  296  ls -r
  297  ls -R
  298  mkdir GIBERT_ALEXIS_MODULE_IMAGE
  299  ls
  300  cd GIBERT_ALEXIS_MODULE_IMAGE/
  301  mkdir Data
  302  mkdir Test
  303  ls
  304  tar -xvf MODULE_IMAGE_V0.tar
  305  cd ..
  306  date
  307  ls -R
  308  cd GIBERT_ALEXIS_MODULE_IMAGE/
  309  ls
  310  cd MODULE_IMAGE/
  311  make mon_module_image.out
  312  ls
  313  sudo apt-get install http://pandoc.org/installing.html
  314  sudo dpkg -i $DEB
  315  cd /
  316  sudo dpkg -i $DEB
  317  cd ~/Downloads/
  318  sudo dpkg -i $DEB
  319  ls
  320  sudo dpkg -i pandoc-2.19.2-1-amd64.deb 
  321  tar xvzf pandoc-2.19.2-linux-amd64.tar.gz --strip-components 1 -C ./~/.local
  322  tar xvzf pandoc-2.19.2-linux-amd64.tar.gz --strip-components 1 -C ~/.local/
  323  ar p pandoc-2.19.2-linux-amd64.tar.gz data.tar.gz | tar xvz --strip-components 2 -C ~/.local
  324  ar p pandoc-2.19.2-1-amd64.deb data.tar.gz | tar xvz --strip-components 2 -C ~/.local
  325  apt-get install texlive
  326  sudo apt-get install texlive
  327  sudo apt-get install haskell-platform
  328  ./mon_module_image.out
  329  cd ..
  330  cd TPC/GIBERT_ALEXIS_MODULE_IMAGE/
  331  ls
  332  cd MODULE_IMAGE/
  333  ./mon_module_image.out
  334  make clean
  335  ls
  336  cd ..
  337  cp MODULE_IMAGE MODULE_IMAGE_V0
  338  mv MODULE_IMAGE MODULE_IMAGE_V0
  339  ls
  340  cp MODULE_IMAGE_V0 ./MODULE_IMAGE_V1
  341  cp MODULE_IMAGE_V0 /MODULE_IMAGE_V1
  342  cp MODULE_IMAGE_V0 MODULE_IMAGE_V1
  343  cd ..
  344  cp ./GIBERT_ALEXIS_MODULE_IMAGE/MODULE_IMAGE_V0 ./GIBERT_ALEXIS_MODULE_IMAGE/MODULE_IMAGE_V1
  345  cd GIBERT_ALEXIS_MODULE_IMAGE/
  346  cp -r MODULE_IMAGE_V0 MODULE_IMAGE_V1
  347  ls
  348  cd MODULE_IMAGE_V1
  349  ls
  350  mv ferrane_module_image.c gibert_module_image.c
  351  mv ferrane_module_image.h gibert_module_image.h
  352  rm mon_module_image.out 
  353  ls
  354  make mon_module_image.out
  355  cat Makefile 
  356  sudo apt-get install nano
  357  nano Makefile 
  358  make mon_module_image.out
  359  nano gibert_module_image.c
  360  make mon_module_image.out
  361  nano gibert_module_image.c
  362  nano gibert_module_image.h
  363  make mon_module_image.out
  364  rm mon_module_image.out 
  365  ls
  366  rm mon_module_image.o 
  367  make mon_module_image.out
  368  nano main.c
  369  make mon_module_image.out
  370  ls
  371  cd ..
  372  ls -r
  373  ls -R
  374  ls -d
  375  ls -Rd
  376  ls -F
  377  ls -RF
  ...
  409  cd GIBERT_ALEXIS_MODULE_IMAGE/
  410  ls
  411  cd MODULE_IMAGE_V0
  412  cd ..
  413  ls
  414  mv ./GIBERT_ALEXIS_TP_IMAGE/NOTEBOOK_TP_IMAGE_GIBERT_ALEXIS.ipynb ~/TPC
  415  ls
  416  GIBERT_ALEXIS_TP_IMAGE/
  417  cd GIBERT_ALEXIS_TP_IMAGE/
  418  ls
  419  mv ./Programs/** ~/TPC/GIBERT_ALEXIS_TP_IMAGE
  420  ls
  421  rm Programs/
  422  rmdir Programs/
  423  mv ./Pictures ~/TPC
  424  ls
  425  cd ..
  426  ls
  427  mv ./*.jpg ./Pictures/
  428  ls
  429  ls -R [GIBERT_ALEXIS]
  430  ls -R [GIBERT_ALEXIS]*
  431  ls -R
  432  ls -Rd
  433  ls -d
  434  ls -dR
  435  ls -R -d
  436  history
\end{verbatim}

    \subsection{MODULE MODULE\_IMAGE\_V1}\label{module-module_image_v1}

\textbf{COPIER-COLLER ICI}

\subsubsection{~(a) le résultat de la commande
date}\label{a-le-ruxe9sultat-de-la-commande-date}

\subsubsection{(b) le code source du programme fichier .c et
.h}\label{b-le-code-source-du-programme-fichier-.c-et-.h}

Première version du fichier \texttt{gibert\_module\_image.c} :

\begin{figure}
\centering
\includegraphics{attachment:TP_IMAGE_NANO9.jpg}
\caption{TP\_IMAGE\_NANO9.jpg}
\end{figure}

\paragraph{\texorpdfstring{Problème lié a la déclaration \texttt{int} à
l'intérieur de la boucle
\texttt{for}}{Problème lié a la déclaration int à l'intérieur de la boucle for}}\label{probluxe8me-liuxe9-a-la-duxe9claration-int-uxe0-lintuxe9rieur-de-la-boucle-for}

\begin{verbatim}
alexis@alexis-VirtualBox:~/TPC/GIBERT_ALEXIS_MODULE_IMAGE/MODULE_IMAGE_V1$ make mon_module_image.out
gcc -o main.o -c main.c -W -Wall -ansi -pedantic
main.c: In function ‘main’:
main.c:15:2: error: C++ style comments are not allowed in ISO C90
   15 |  //lecture sur l'entréee standard
      |  ^
main.c:15:2: note: (this will be reported only once per input file)
main.c:16:10: error: redeclaration of ‘i’ with no linkage
   16 |  for(int i=0;i<10;i++) scanf("%d",&tab_niveau[i]);
      |          ^
main.c:13:10: note: previous declaration of ‘i’ was here
   13 |  int res,i;
      |          ^
main.c:16:2: error: ‘for’ loop initial declarations are only allowed in C99 or C11 mode
   16 |  for(int i=0;i<10;i++) scanf("%d",&tab_niveau[i]);
      |  ^~~
main.c:16:2: note: use option ‘-std=c99’, ‘-std=gnu99’, ‘-std=c11’ or ‘-std=gnu11’ to compile your code
make: *** [Makefile:8: main.o] Error 1
\end{verbatim}

\begin{quote}
\textbf{Solution :} Le Makefile semble rétissant a la déclaration à
l'intérieur de boucle. Même si c'est moins optimisé, il m'a donc fallut
monter la déclaration au niveau des déclations générales de la fonction.
\end{quote}

\paragraph{Problème lié aux
commentaires}\label{probluxe8me-liuxe9-aux-commentaires}

\begin{verbatim}
alexis@alexis-VirtualBox:~/TPC/GIBERT_ALEXIS_MODULE_IMAGE/MODULE_IMAGE_V1$ make mon_module_image.out
gcc -o main.o -c main.c -W -Wall -ansi -pedantic
main.c: In function ‘main’:
main.c:15:2: error: C++ style comments are not allowed in ISO C90
   15 |  //lecture sur l'entréee standard
      |  ^
main.c:15:2: note: (this will be reported only once per input file)
make: *** [Makefile:8: main.o] Error 1
\end{verbatim}

\begin{quote}
\textbf{Solution :} Le Makefile est rétissant au commentaires monoligne
avec la syntaxe \texttt{//}. Ainsi, pour mettre des commentaires il
\textgreater{}faut donc obligatoirement précéder le commentaire de
\texttt{/*} et le terminer par \texttt{*/}
\end{quote}

\paragraph{\texorpdfstring{Problème lié au \texttt{log2()} contenu dans
la bibliothèque
\texttt{math.h}}{Problème lié au log2() contenu dans la bibliothèque math.h}}\label{probluxe8me-liuxe9-au-log2-contenu-dans-la-bibliothuxe8que-math.h}

\begin{verbatim}
alexis@alexis-VirtualBox:~/TPC/GIBERT_ALEXIS_MODULE_IMAGE/MODULE_IMAGE_V1$ make mon_module_image.out
gcc -o main.o -c main.c -W -Wall -ansi -pedantic
gcc -o mon_module_image.out mon_module_image.o main.o
/usr/bin/ld: mon_module_image.o: in function `afficher_image_codee':
gibert_module_image.c:(.text+0x55): undefined reference to `log2'
/usr/bin/ld: gibert_module_image.c:(.text+0x66): undefined reference to `log2'
collect2: error: ld returned 1 exit status
make: *** [Makefile:2: mon_module_image.out] Error 1
\end{verbatim}

\textbf{Test 1 :} J'ai utilisé la commande
\texttt{sudo\ apt-get\ update} puis la commande
\texttt{sudo\ apt-get\ upgrade} pour mettre a jour tous les packages sur
Ubuntu.

\begin{quote}
\textbf{Résultat :} La même erreur survient.
\end{quote}

\begin{verbatim}
alexis@alexis-VirtualBox:~/TPC/GIBERT_ALEXIS_MODULE_IMAGE/MODULE_IMAGE_V1$ make mon_module_image.out
gcc -o mon_module_image.out mon_module_image.o main.o
/usr/bin/ld: mon_module_image.o: in function `afficher_image_codee':
gibert_module_image.c:(.text+0x55): undefined reference to `log2'
/usr/bin/ld: gibert_module_image.c:(.text+0x66): undefined reference to `log2'
collect2: error: ld returned 1 exit status
make: *** [Makefile:2: mon_module_image.out] Error 1
\end{verbatim}

\textbf{Test 2 (Solution) :} Après avoir analysé le Makefile et les
différentes options j'ai compris que l'erreur venait - de l'option
\texttt{-lm} a rajouter dans les fichiers nécessitant la librairie
math.h - de l'option \texttt{-ansi} puisque elle force le standard "c89"
alors que la fonction \texttt{log2()} appartient au standard "c99"

ainsi il faut modifier le makefile de la manière suivante :

\begin{verbatim}
  GNU nano 4.8                                  Makefile                                             
mon_module_image.out: mon_module_image.o main.o
        gcc -o mon_module_image.out mon_module_image.o main.o -lm -std=c99

mon_module_image.o: gibert_module_image.c
        gcc -lm -std=c99 -c gibert_module_image.c -o mon_module_image.o -W -Wall -pedantic

main.o: main.c gibert_module_image.h
        gcc -o main.o -c main.c -std=c99 -W -Wall -pedantic

clean:
        rm -rf *.o
\end{verbatim}

\textbf{Soit le code source final du fichier
\texttt{gibert\_module\_image.c} :}

\begin{verbatim}
#include <stdio.h>
#include <stdlib.h>
#include <math.h>
#include "gibert_module_image.h"

/* AUTEUR : ALEXIS GIBERT                       */
/* DATE CREATION : 30/10/2022                      */
/*-------------------------------------------------*/

/* DEFINITIONS DES FONCTIONS déclarées dans le .h */

void test_prog(void)
{
    printf("TEST DU MODULE : GIBERT_MODULE_IMAGE\n");
}

int afficher_image_codee(int niveau){
  int nbl,nbc,pxlc=0,i,j;
  if(niveau<256){
    if((int)log2(niveau)==log2(niveau)){
      scanf("%d",&nbl);
      scanf("%d",&nbc);
      for(i=0;i<nbl;i++){
        for(j=0;j<nbc;j++){
          scanf("%d",&pxlc);
          printf("%5d",((pxlc*niveau)/256));
        }
        printf("\n\r");
      }
    }
    else return 0;
  }
  else return -1;
  return 1;
}
\end{verbatim}

\textbf{Soit le code source final du fichier
\texttt{gibert\_module\_image.h} :}

\begin{verbatim}
#ifndef MODULE_IMAGE_H_INCLUS     /*--  Inclusion conditionnelle --> si pas déjà inclus           */
#define MODULE_IMAGE_H_INCLUS     /*--  alors créer la constante symbolique MODULE_IMAGE_H_INCLUS */

/* AUTEUR : ALEXIS GIBERT                       */
/* DATE CREATION : 30/10/2022                      */
/*-------------------------------------------------*/

/* DECLARATIONS DES FONCTIONS */

void test_prog(void);
int afficher_image_codee(int niveau);

#endif
\end{verbatim}

\textbf{Soit le code source final du fichier \texttt{main.c} :}

\begin{verbatim}
#include <stdio.h>
#include <stdlib.h>

#include "gibert_module_image.h"

/* AUTEUR : GIBERT ALEXIS                       */
/* DATE CREATION : 30/10/2022                      */
/*-------------------------------------------------*/

int main(void)
{
    int niveau,res;

    scanf("%d",&niveau);

    printf("\n\rTest avec niveau [%d]\n\r",niveau);
    res=afficher_image_codee(niveau);
    if(res==-1) printf("Erreur 1 : le niveau de gris est supérieur à 256\n\r");
    else if(res==0) printf("Erreur 2 : le niveau de gris n’est pas une puissance de 2\n\r");
    else if(res==1) printf("Traitement effectué\n\r");
    else printf("Erreur 3 : Valeur inconnue renvoyée par la fonction\n\r");
    return EXIT_SUCCESS;
}
\end{verbatim}

\subsubsection{(c) la compilation et son
résultat}\label{c-la-compilation-et-son-ruxe9sultat}

\begin{verbatim}
alexis@alexis-VirtualBox:~/TPC/GIBERT_ALEXIS_MODULE_IMAGE/MODULE_IMAGE_V1$ make mon_module_image.out
gcc -lm -std=c99 -c gibert_module_image.c -o mon_module_image.o -W -Wall -pedantic
gcc -o main.o -c main.c -std=c99 -W -Wall -pedantic
gcc -o mon_module_image.out mon_module_image.o main.o -lm -std=c99
\end{verbatim}

\subsubsection{(d) les résultats des tests
effectués}\label{d-les-ruxe9sultats-des-tests-effectuuxe9s}

Après déplacement du fichier \texttt{image1.dat} dans le répertoire
\texttt{\textasciitilde{}/TPC/GIBERT\_ALEXIS\_MODULE\_IMAGE/Data/image1.dat}
on obtient les résultats suivants :

\begin{verbatim}
alexis@alexis-VirtualBox:~/TPC/GIBERT_ALEXIS_MODULE_IMAGE/MODULE_IMAGE_V1$ cat save ~/TPC/GIBERT_ALEXIS_MODULE_IMAGE/Data/image1.dat | ./mon_module_image.out

Test avec niveau [0]
Erreur 2 : le niveau de gris n’est pas une puissance de 2

alexis@alexis-VirtualBox:~/TPC/GIBERT_ALEXIS_MODULE_IMAGE/MODULE_IMAGE_V1$ cat niveau ~/TPC/GIBERT_ALEXIS_MODULE_IMAGE/Data/image1.dat | ./mon_module_image.out

Test avec niveau [2]
    0    0    0    0    0    0    0    0
    0    0    0    0    0    1    1    0
    0    0    0    0    1    1    0    0
    0    0    0    0    0    1    1    0
    0    0    0    0    0    1    0    0
    0    0    0    0    0    1    0    0
    0    0    0    0    1    1    1    0
    0    0    0    0    0    0    0    0
Traitement effectué

alexis@alexis-VirtualBox:~/TPC/GIBERT_ALEXIS_MODULE_IMAGE/MODULE_IMAGE_V1$ cat niveau ~/TPC/GIBERT_ALEXIS_MODULE_IMAGE/Data/image1.dat | ./mon_module_image.out

Test avec niveau [4]
    0    0    0    0    0    0    0    0
    0    0    0    0    0    2    2    0
    0    0    0    0    2    2    0    0
    0    0    0    0    0    2    2    0
    0    1    0    0    0    2    0    0
    0    1    0    0    0    2    0    0
    0    1    1    0    2    2    2    0
    0    0    0    0    0    0    0    0
Traitement effectué

alexis@alexis-VirtualBox:~/TPC/GIBERT_ALEXIS_MODULE_IMAGE/MODULE_IMAGE_V1$ cat niveau ~/TPC/GIBERT_ALEXIS_MODULE_IMAGE/Data/image1.dat | ./mon_module_image.out

Test avec niveau [15]
Erreur 2 : le niveau de gris n’est pas une puissance de 2

alexis@alexis-VirtualBox:~/TPC/GIBERT_ALEXIS_MODULE_IMAGE/MODULE_IMAGE_V1$ cat niveau ~/TPC/GIBERT_ALEXIS_MODULE_IMAGE/Data/image1.dat | ./mon_module_image.out

Test avec niveau [32]
    0    0    0    0    0    0    0    0
    0    4    4    0    0   18   18    0
    0    4    4    0   18   18    0    0
    0    0    0    0    0   18   18    0
    0   11    0    0    0   21    0    0
    0   11    0    0    0   21    0    0
    0   11   11    0   21   21   21    0
    0    0    0    0    0    0    0    0
Traitement effectué

alexis@alexis-VirtualBox:~/TPC/GIBERT_ALEXIS_MODULE_IMAGE/MODULE_IMAGE_V1$ cat niveau ~/TPC/GIBERT_ALEXIS_MODULE_IMAGE/Data/image1.dat | ./mon_module_image.out

Test avec niveau [260]
Erreur 1 : le niveau de gris est supérieur à 256
\end{verbatim}

    


    % Add a bibliography block to the postdoc
    
    
    
    \end{document}
